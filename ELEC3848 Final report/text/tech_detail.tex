
\section{Techinical Details}
\subsection{Communication among Jetson and Arduino boards}
We defined our customized protocol as shown in Table \ref{tab:com_protocol}.
\begin{table}[h]
    \centering
    \begin{tabular}{|p{1in}|p{1in}|p{3in}|}
        \hline 
        Offset (Byte) & Length (Byte) & Content \\ \hline 
        0 & 2 & SOF (0x807F) \\ \hline 
        2 & 1 & Packet ID \\ \hline
        3 & 2 & Payload Size \\ \hline
        5 & N & Payload Data \\ \hline
        5+N & 2 & CRC16 \\ \hline 
        7+N & 2 & EOF (0x8080) \\ \hline 
    \end{tabular}
    \caption{Customized data protocol}
    \label{tab:com_protocol}
\end{table}

The pack and send function is demonstrated in pseudocode in Lists below.

\begin{lstlisting}
// Pack and send data to TX 
procedure Send
    buf[0 : 1] = SOF
    buf[2] = id
    buf[3 : 4] = size
    buf[5 : 5+size-1] = data
    crc = crc16(buf[2 : 5+size-1])
    buf[5+size : 6+size] = crc
    buf[7+size : 8+size] = EOF
    send buf to serial
endprocedure 
\end{lstlisting}

\begin{lstlisting}
// Receive data from serial buffer
procedure Receive
    append serial buffer to buf
    sof_idx = -1
    eof_idx = -1
    for i = 0 to eff_len(buf) - 2
        if (buf[i:i+1] == SOF)
            sof_idx = i
        else if (buf[i:i+1] == EOF) 
            unpack buf[sof_idx+2 : eof_idx-1]
        endif
    endfore 
endprocedure 
\end{lstlisting}

\begin{lstlisting}
// unpack data into the respective recv_dst
function unpack (data)
    id = data[0]
    size = data[1 : 2]
    calc_crc = crc16(data[0 : 2+size])
    tran_crc = data[3+size : 4+size]
    if (tran_crc == calc_crc)
        recv_dst[id] = data[3 : 2+size]
        if recv_callback[id] != NULL
            recv_callback[id]()
        endif
    endif
endfunction
\end{lstlisting}

We coded the following function to assign the destination pointer to \lstinline|recv_dst| array: 
\begin{lstlisting}
int communication_register_recv
(communication_t comm, uint8_t pkt_id, void* data_ptr, 
 size_t len, recv_callback_t callback)
\end{lstlisting}

\lstinline|callback| if called if packet of this ID is correctly received. 
In this way, by using the following code
\begin{lstlisting}
void forward_roboarm_cmd() {
    communication_send(&com_S3, ROBOARM_CMD_ID, 
                       &fwd_roboarm_cmd, sizeof(fwd_roboarm_cmd));
}
void forward_sensor_fdbk() {
    communication_send(&com_S2, SENSOR_FDBK_ID, 
                       &fwd_sensor_fdbk, sizeof(fwd_sensor_fdbk));
}
void chassis_setup() {
    communication_register_recv(&com_S2, ROBOARM_CMD_ID, &fwd_roboarm_cmd, 
            sizeof(fwd_roboarm_cmd), (recv_callback_t)&forward_roboarm_cmd);
    communication_register_recv(&com_S3, SENSOR_FDBK_ID, &fwd_sensor_fdbk, 
            sizeof(fwd_sensor_fdbk), (recv_callback_t)&forward_sensor_fdbk);
    ...
\end{lstlisting}
We can easily realize the data switching between the two (or even more, if needed) Arduino boards.
 
% The SOF (start of frame) and EOF (end of frame) 

\subsection{Protocol}
With the communication module above, data can be easily exchanged by sending and receiving the packed structure. There are 4 types of packets: chassis command, chassis feedback, roboarm command, and sensor feedback. 

In Arduino code, they are defined as


\subsection{DC Motor speed control with PID}

To measure the wheel speed, following macro function is coded to assign the motor and register encoder interrupt.
\begin{lstlisting}
#define MOTOR_ASSIGN(NAME, PIN_1, PIN_2, PIN_PWM, PIN_A, PIN_B)     \
void isr_##NAME();                                                  \
struct motor_device NAME = {                                        \
    .pin_1 = PIN_1, .pin_2 = PIN_2, .pin_pwm = PIN_PWM,             \
    .pin_ecd_A = PIN_A, .pin_ecd_B = PIN_B, .ecd_isr = isr_##NAME   \
};                                                                  \
void isr_##NAME(void) {                                             \
    if(digitalRead(NAME.pin_ecd_B)) NAME.data.total_ecd++;          \
    else NAME.data.total_ecd--;                                     \
}
\end{lstlisting}

PID library developed by DJI Robomaster is borrowed 
and applied to control the motors, with the four speed of mecanum as reference
and encoder measured speed as feedback. The speed unit is is in RPM, and PID parameters used are
\begin{table}[H]
    \centering
    \begin{tabular}{|p{2in}|p{2in}|}
       \hline 
       Parameter  & Value  \\ \hline 
       $k_p$  & 0.8 \\ \hline 
       $k_i$ & 0.1 \\ \hline 
       $k_d$ & 0 \\ \hline 
       Maximum input error & 0 \\ \hline 
       Maximum output & MOTOR\_DUTY\_MAX (255) \\ \hline 
       Integral limit & MOTOR\_DUTY\_MAX / 5 (51) \\ \hline
    \end{tabular}
    \caption{Chassis wheel motor PID params}
    \label{tab:chassis_pid_params}
\end{table}

\subsection{Line following}
The raw line tracker value is stored in a 8-bit variable as
\begin{lstlisting}
value = (digitalRead(sensor->pin_3) << 3) |
        (digitalRead(sensor->pin_2) << 2) |
        (digitalRead(sensor->pin_1) << 1) |
        (digitalRead(sensor->pin_0));
\end{lstlisting}

And an error is calculated based this value, if the two outer sensors see black, 
then use the outer sensor's bias, otherwise, use the two inner sensors.  
\begin{lstlisting}
error = (((value & 0b1000) >> 3) - ((value & 0b0001))) * 3;
if (error == 0)
   error = ((value & 0b0100) >> 2) - ((value & 0b0010) >> 1);
\end{lstlisting}

Use this error as feedback, a PID controller is applied. The reference is always 0, and output is used to get the left and right side wheel speed as 
\begin{lstlisting}
left = 1 - pid_out;
right = 1 + pid_out;
\end{lstlisting}

\subsection{Roboarm Calculate}
For our 2-DoF roboarm, Joint-1 controls the backward-forward of the arm, which also change the height of the Joint-2. Joint-2 controls the pitch angle of the clamp. Thus the roboarm can be controlled via a height and an angle.

Upon receiving the height, the angle $\theta$ of Join-1 servo is computed by 
\begin{lstlisting}
h = height - H_OFFSET;
if (h > ARM_LEN)
    h = ARM_LEN;
theta = degrees( acos(h / (ARM_LEN + 0.1)) );
\end{lstlisting}

\subsection{Clamp close}
When the ``close'' flag in roboarm command is 1, the clamp will close. It will first close util the 
micro switch assert, and then close for a further 20 degrees to make the grabbing tight enough.
This is implemented with the help of ``clamp\_tight'' flag. Each loop it will do
\begin{lstlisting}
if switch touched
    clamp_deg = clamp_deg - 1
    servo.set_angle(clamp_deg)
    clamp_tight = false
else if clamp_tight == false
    clamp_deg = clamp_deg - 20
    servo.set_angle(clamp_deg)
    clamp_tight = true
endif
\end{lstlisting}